\section{Literature Review} \label{litreview}
\begin{sectionplan}
Introduction - Tell the person what you're going to tell them. Talk about what
this chapter is about with reference to the various subsections of the chapter.
\end{sectionplan}

\subsection{Parallelisation Methods}
\begin{sectionplan}
     What are the different means of parallelisation? Which ones are being
studied here?
\end{sectionplan}

\subsubsection{Single Instruction Multiple Data (SIMD)}
\begin{sectionplan}
	SIMD - hard to use and optimise with. Good for speeding up single threaded
	operations if a use case is found.
\end{sectionplan}

\begin{roughwork}

The first method  I’ve looked at was SIMD. SIMD instructions are assembly
instructions where instead of processing data using normal registers, SIMD
instructions uses special registers that are several times bigger than a typical
register. This allows the processor to compute a given instruction over each
part of the register in parallel. As an example, The arm ADD insturction can
be contrasted with its corresponding vector instruction UQADD8 which adds 4
8bit numbers with 4 other 8bit numbers. Performance gains using this method
can be substantial however only up to a point. More performance requires larger
registers which takes a long time to architect and standerdize in a CPU. The
next method is more scalable.

\end{roughwork}

\subsubsection{Multithreading / Multiprocessing}
\begin{sectionplan}
     Multi-threading / multi-processing - hard to work around in a compiler but
commonly available.
\end{sectionplan}

\begin{roughwork}

	With multicore parallism, its possible to have multiple programs exectute at the
	same time. Each process can have its own call stack and its own view of memory.
	This gives a programmer tremendous freedom with the things that can be done in
	parallel. The down side here is the difficulty in creating a reliable multi-
	threaded program. Unfortunatlely, if one is coverting a sequential program it
	a parallel one, it is usually necessary to re-architect significant portions of
	it in order to better fit this processing model. The benefit is the theoretical
	promise of near-linear performance improvement with each additional core in the
	cpu. So, for example, if a strongly parallel program can run in 1 second on one
	thread then in theory it should run in a quarter of a second on four threads. In
	other words, a four times speed up. The next method takes this approach to the
	next level.

\end{roughwork}


\subsubsection{General Purpose Computing On GPUs (GPGPU)}
\begin{sectionplan}
     large scale heterogeneous computing (gpu) - Imposes many limitations along
with a very large overhead that makes it useless for smaller - scale workloads.
Availability varies.
\end{sectionplan}

\begin{roughwork}

	It is possible to use graphics cards for general purpose computing, called
	GPGPU. It is conceptually like programming a specialised CPU that has
	potentially thousands of cores. This might seem like an obvious advantage over
	programming a multicore cpu however a program that runs on the GPU has many
	peculiarities and restrictions place on it that it requires a programmer to re-
	architect a program even further in order to fit this programming model, as
	compared to the multicore case. As an example, memory latency between a gpu and
	main memory is usually so big that it makes the GPU significantly worse at for
	processing small amounts of data.

\end{roughwork}

\subsubsection{Very Large Instruction Word (VLIW)}
\begin{sectionplan}
     VLIW can be used to make any program parallel - general purpose VLIW cpu's
not available for this purpose, mostly use in DSP's - esoteric.
\end{sectionplan}
\cite{fisher_parallel_2004}

\subsection{Methods of Parallelising A Compiler} \label{compiler_parallel_methods}
\begin{sectionplan}
    Mention the different ways to architect a parallel compiler program using
aforementioned parallel processing methods.
\end{sectionplan}
\cite{hillis_data_1986, gross_parallel_1989, jena_design_2018, baer_model_1977}

\begin{roughwork}

	So, after a survey of what parallel processing methods are available, we can
	take a look at what approaches have been thought about in the past. I found
	that a paper from 1989 which the main approaches into the three points on this
	slide. The first approach is splitting up the input and processing each chunk
	individually. Many compilers divide up data at the file system level where each
	file of source code can be compiled seperately from other files it may depend
	on. Ideally, in a truly parallel compiler, each file could be split up further
	into even smaller chunks with each chunk being processed independantly. This
	approach is great but it causes issues that I will describe later. Computation
	partitioning on the other hand assumes that a series of sequential computations
	can be divided up and processed in parallel with results of each computation
	being joined joined together at the end. The final method is pipeling which
	is similar to computation partitioning. Pipelining involves dividing up a
	computation into some number of phases where each phase depends on the output
	from the previous phase. A complier can conveniently be divided up into such
	a pipline where, for example, the lexer, parser and semantic analyzer can be
	potentially executed simultaneusly. For my project I will focus predominantly on
	the data paritioning method and I will elaborate on my reasons for this decision
	in my final report.

\end{roughwork}

\subsection{Lexing}
\begin{sectionplan}
    What is involved in lexing. Mention DFA / Finite state machine.
\end{sectionplan}
\cite{pai_t_systematic_2020, barve_parallel_2014, barve_parallel_2012, barve_improved_2015}

\subsubsection{Naive Multithreaded}
\cite{barenghi_parallel_2015, mytkowicz_data-parallel_2014}

\subsubsection{Speculative Prescanning}
\cite{li_plex_2021}

\subsubsection{Novel Approaches}
\cite{sinya_simultaneous_2013, lin_accelerating_2013}

\subsubsection{Vectorization}
\cite{wang_hyperscan_2019}

\subsection{Parsing}
\begin{sectionplan}
    What factors generally influence parallel parsing methods?
\end{sectionplan}
\cite{mark_thierry_vandevoorde_parallel_1988}

\subsubsection{CYK}
\cite{skrzypczak_parallel_nodate}
\subsubsection{OPG Parsing}
\cite{barenghi_parallel_2015, li_associative_2023}
\subsubsection{LLP}
\cite{robin_voetter_parallel_2021}
\subsubsection{LRP}

\subsection{Semantic Analysis}
\begin{sectionplan}
    \begin{itemize}
        \item Identify what is in semantic anlysis
        \item What occurs during semantic analysis?
        \begin{itemize}
            \item Correcting parse tree if only subset of possible grammar productions are accepted
            \item Symbol table generation
        \end{itemize}
        \item Ways of navigating an AST in parallel
    \end{itemize}
\end{sectionplan}
