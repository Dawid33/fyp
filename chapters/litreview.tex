\chapter{Literature Review} \label{litreview}
\begin{sectionplan}
Introduction - Tell the person what you're going to tell them. Talk about what
this chapter is about with reference to the various subsections of the chapter.
\end{sectionplan}

As part of this literature review I research existing techniques used to create
parallel compilers. This form of study has allowed me to better understand how
parallel compilers work and how to approach implementing one. Due to a large
amount of literture on the topic, I only review as many exisiting approaches
as time allows. An important reference used througout the literature review is by
\cite{scott_programming_2015}  which introduces various parts of a compiler at
a high-level.
\newline \newline
Section \ref{parallelisation_methods} describes general methods of executing
code in parallel.
\newline \newline
Section \ref{compiler_parallel_methods} describes different ways of designing a compiler
such that parts of it can be executed simulatneously.
\newline \newline
Section \ref{lit_review_lexing} describes various algorithms to perform lexing
in parallel. \cite{pai_t_systematic_2020} describes a literature review of parallel lexing
approaches.
\newline \newline
Section \ref{lit_review_parsing} describes various algorithms to perform parsing
in parallel.
\newline \newline
Section \ref{lit_review_analysis} describes various algorithms to perform semantic
analysis in parallel.
\newline \newline

\section{Parallelisation Methods} \label{parallelisation_methods}
There exist various hardware solutions for executing code in parallel.
The choice of parallel processing methods strongly influences the ease of
implementation and performance of the final program. This makes it worthwhile to
research and analyze the various forms so parallelisation so as to make informed
decisions about how they are used in a parallel program.

\subsection{Single Instruction Multiple Data (SIMD)}
\gls{simd} instructions are assembly instructions, that can process more data
at the same time than typical assembly instructions. Instead of processing data
with normal registers, \gls{simd} instructions use special registers that are
several times bigger than a typical register (\textbf{find citation}). This allows
the processor to compute a given instruction over each part of the register in
parallel. These instructions are also known as vector instructions since they
process lists or vectors of data. As an example, the arm ADD insturction can be
contrasted with its corresponding vector instruction UQADD8 which adds 4 8bit
numbers with 4 other 8bit numbers (\textbf{cite amd's website}).

This form of parallelisation mimics the sequential approach as vector
instructions do multiple sequential instructions at the same time. This makes
it possible for compilers to automatically utilise these instructions to speed
up sequential programs in a process called vectorisation. Vector instructions
can also be manually written by the programmer in order to improve performance.
The performance gains using this method are clearly demonstrated with simdJSON
(\textbf{citation needed for simdjson}). Although using vector instructions can
be significantly faster than the sequential case, this performance benefit is
only up to a point. Increasing performance requires larger registers which takes
a long time to architect and standardize in a \gls{cpu}. The methods described
next scale better in this regard at the cost of potentially having a more
complex implementation.

\subsection{Multithreading/Multiprocessing}
Multicore parallelism makes it possible to have multiple programs exectute at
the same time. Each process can have its own call stack and its own view of
memory. This gives a programmer tremendous freedom with the things that can be
implemented in parallel. The down side is the difficulty in creating reliable
multi-threaded programs. Managing shared memory is difficult and can lead to
very complex bugs.

When coverting a sequential program to a \gls{data_parallel} one, it is usually
necessary to re-architect significant portions of it in order to better fit
this processing model. The advantage to this is the potential of near-linear
performance improvement that scales with additional available cores. For
example, if a strongly parallel program can run in 1 second on one thread then
in it should run in a quarter of a second on four threads. In other words, a
four times speed up. In practice there is overhead which prevents prefectly
linear performance improvement.


\subsection{General Purpose Computing on GPUs}
It is possible to use graphics cards for general purpose computing, called
\gls{gpgpu}. It is conceptually like programming a specialised \gls{cpu} that has
potentially thousands of cores. This might seem like an obvious advantage over
programming a multicore \gls{cpu} however a program that runs on the \gls{gpu} has
many peculiarities and restrictions placed on it that it requires a programmer
to re-architect a program even further in order to fit its programming model, as
compared to the multicore case. As an example, memory latency between a \gls{gpu}
and main memory is usually so big that it makes the \gls{gpu} significantly worse
at for processing small amounts of data.

\subsection{Very Large Instruction Word}
\begin{sectionplan}
     VLIW can be used to make any program parallel - general purpose VLIW cpu's
not available for this purpose, mostly use in DSP's - esoteric.
\end{sectionplan}
\cite{fisher_parallel_2004}

\section{Methods of Parallelising A Compiler} \label{compiler_parallel_methods}
\begin{sectionplan}
    Mention the different ways to architect a parallel compiler program using
aforementioned parallel processing methods.
\cite{hillis_data_1986, gross_parallel_1989, jena_design_2018, baer_model_1977}
\end{sectionplan}

\cite{gross_parallel_1989} outlines three main approaches for parallising a
compiler. The first approach is splitting up the input and processing each chunk
individually. Many compilers divide up data at the file system level where each
file of source code can be compiled seperately from other files it may depend
on. Although this approach is already \gls{data_parallel}, it can be taken even
further where each file is split up into even smaller chunks with each chunk
being processed independantly.

Computation partitioning on the other hand assumes that a series of sequential
computations can be divided up and processed in parallel with results of
each computation being joined together at the end.

The final method is pipeling which is similar to computation partitioning.
Pipelining involves dividing up a computation into some number of phases
where each phase depends on the output from the previous phase. A complier can
conveniently be divided up into such a pipline where, for example, the lexer,
parser and semantic analyzer can be potentially executed simultaneously. 

\section{Lexing} \label{lit_review_lexing}

Lexing is the process of process of removing unneccesary characters from the
source code and grouping characters of interest together into tokens. This stage
emits a list of tokens that are passed on to the parser. The purpose of lexing
is to simplify the task of the parser by reducing the size of the input. Lexing
is sometimes  called tokenisation or lexical analysis. It is performed by a
lexer, sometimes called a tokenizer or lexical analyzer. Its task of finding
groupings of characters can be performed by using \glspl{reg_exp} that
define what sequence of characters need to be found in the source code. The use
of \glspl{reg_exp} means that a lexer can be described as a finite state
machine.

A \gls{fsm} is a model of an abstract machine that can be in one of a finite
number of states at any given time. A \gls{fsm} works by accepting input symbols
which potentially cause the state of the \gls{fsm} to change according to a set
of state transitions that define the \gls{fsm}. Finite state machines can be
be described as either a \gls{nfa} or as a \gls{dfa}. In a parallel lexer where
the output in one thread may depend on the output of another thread, it is
undesirable to have different results based on the same input as it leads to
bugs and an unpredictable output for a given input. This situation can occur
in a \gls{nfa} which makes it non-trivial to design a parallel lexer around an
\gls{nfa}. The determinstic property of a \gls{dfa} enusures that the that the
changes in state incurred from a given input and inital state will always be
the same for that input. As such, structuring a lexer as a \gls{dfa} can ensure a
consistent and repeatable computation for a given chunk of source code.

A common approach to multi-threaded lexing involves structuring the lexer as
a \gls{dfa}, splitting the input into some number of parts, lexing each part on
its own thread and joining up the results at the end. Due to the inherent data
dependancy between state transitions in a \gls{dfa}, the initial state for a
chunk of code depends on the output computed from lexing the code just before
it. This causes exists an issue of determining the initial start state of all
the lexers besides the one responsible for computing the first part of the code.
A solution to this problem is through speculative simulation where an algorithm
speculates on the unkown initial states of the \gls{dfa}s.

\subsection{Speculative Simulation of DFA}

One approach to speculative simulation is the brute force method of computing a
\gls{dfa} for every possible initial state  Such a lexer starts by splitting its
input into chunks and processing each chunk independantly on a separate thread.
Once all threads are finished lexing, the outputs from each thread are checked,
starting from the first chunk of code, such that the finishing state of the
lexer corresponds to the initial state of the lexer that computed the subsequent
chunk of code. Computing a \gls{dfa} in this way relies on the  associativity of
state transition functions as described in the parallel prefix sum algorithm in
\cite{hillis_data_1986}. This algorithm is explained more formally wth both a
message passing and a shared memory implementation by \cite{holub_parallel_2009}
and a variation of it in a cloud computing environment is shown by
\cite{ko_speculative_2012}. The need to compute the output for every possible
initial state can result in a large number of unneccesary computations which can
significantly impact perfromance for \gls{dfa}s with a large number of states.

\cite{barenghi_parallel_2015} optimise this by using a heuristic whereby the
source code is split into chunks that start with symbols that have known and
consistent initial states. These symbols are found by scanning ahead when
the source code is being split until such a symbol is found. The symbols
that a chunk of code can start with are langauge dependant. If a symbol is
allowed to appear in a lexeme, such a string or comment, then multiple initial
states must be additionally computed. 

Instead of using a language specific heuristic,
\cite{mytkowicz_data-parallel_2014} aim optmise speculative simulation for
any given \gls{dfa}. The core algorithm is very similar to the one previously
described. It begins with the assumption of needing to compute the \gls{dfa} for
every possible initial state the input can be in. Instead of using a language
specific heuristic like \cite{barenghi_parallel_2015}, which is determined ahead
of time, \cite{mytkowicz_data-parallel_2014} define a convergence algorithm
which reduces the number of states that need to be computed at runtime. It
follows from the observation that many states transition to the same state on
a given input symbol. Once this convergence of state occurs, looking up the
state transisition for each state becomes redundant as it is going to be the
same for subsequent input symbols. By factoring out these common states, the
number of actual state transitions that need to be computed drops significantly
for most \gls{dfa}s, especially structured and non-adverserial ones. Heuristics
are used to determine when to check for a  convergence of states. The ahead of
time heuristic as well as this convergence algorithm are fundamentally similar
however the solution proposed by \cite{mytkowicz_data-parallel_2014} is more
generally applicable by not being tied to a specific language.

\cite{mytkowicz_data-parallel_2014} additionally utilise \gls{simd} instructions
in order to perform the state transitions for a given set of states and
transition function. This parallelises an important part of the algorithm and
significantly improves performance. An technique called range coalescing is
used to reduce the number of memory addresses accessed when computing state
transitions for a large number of states (\textbf{possibly elaborate}).

\cite{zhao_--fly_2015} implements a convergence algorthim similar to the
one described by  \cite{mytkowicz_data-parallel_2014}. (\textbf{elaborate on
additional optmizations by zhao})

Another method of speculative simulation attempts to simply guess the initial
state of the \gls{dfa}s, possibly with a way to back track or validate
its result in case of an incorrect guess. \cite{luchaup_multi-byte_2009,
luchaup_speculative_2011} implements a speculative method which simply guesses
the unknown inital \gls{dfa} state. This optmization works due to its specifc
use in intrusion decection systems where a \gls{dfa} spends most of its time
in a small number of states which can be guessed with sufficient accuracy to
outweight the cost of validation in case of failure.

\subsection{Prescanning} \label{lit_prescanning}

\cite{bernecky_spmdsimd_2003} describes a multi-pass parallel APL tokenizer
written in APL which performs several scanning phases in order to pre-process
source code as well as find strings, comments and identifiers, among other
tokens, in distinct passes over the source code. It is not table driven and is
instead specialised for APL. Since the computational complexity of the compiler
is high and its performance is not evaluated by the authors, it remains unknown
if is an improved approach to tokenizing APL.

A variation of the heuristic apporach by \cite{barenghi_parallel_2015} is
implemented by \cite{li_plex_2021} that generalizes it and makes it langauge
independant. It builds on work from \cite{sinya_simultaneous_2013} and
\cite{zhao_--fly_2015} to perform a backtrack-free prescanning phase by
generating and executing a \gls{dfa} based on the lexical grammar that determines
the context for \gls{dfa} in a subsequent tokenisation phase. (\textbf{its a
tough one with tough prior work, needs elaboration})


\subsection{Other Approaches}

\begin{roughwork}
\cite{sinya_simultaneous_2013} propose a novel type of finite automata called
\gls{sfa}.
\newline \newline
Mention \cite{lin_accelerating_2013, wang_hyperscan_2019,
li_plex_2021, asthagiri_associative_1992} Go and look at references cited in
\cite{zhao_--fly_2015}
\newline \newline
Could mention these but the fella doesn't do anything novel (and his
grammar is questionable) \cite{barve_parallel_2014, barve_parallel_2012,
barve_improved_2015}
\end{roughwork}

\section{Parsing} \label{lit_review_parsing}
\begin{sectionplan}
    What factors generally influence parallel parsing methods?
\end{sectionplan}

\cite{mark_thierry_vandevoorde_parallel_1988, alblas_bibliography_1994}

Parsing is the second canoncial stage of a generic compiler.
\cite{scott_programming_2015} describes the goal of a parser is take a list
of tokens from a lexer and structure them as a graph. This structure makes it
easier to navigate and manipulate the source code using concepts defined by the
language like variables, functions and control structures. The graph strucuture
created by a parser is typically called a syntax tree or a parse tree. This
structure is created according to a parsing grammar that defines how it should
be built. The rules that govern how tokens in a parsing grammar should be
processed in order to validate the langauge and build a syntax tree representing
it are called production rules. The tokens A parser might addtionally process a syntax
tree as it is getting built, or during a separate susequent processing step,
which produces an AST. This datastructure is later analysed during the semantic
analysis phase which is described in Section \ref{lit_review_analysis}.

There exist different families of parsers which are defined by similarities  in
the method of parsing. LL parsers read the input tokens from left to right and


\subsection{CYK Parsing}
\cite{skrzypczak_parallel_nodate}

\subsection{Parallel LL Parsing}
\cite{robin_voetter_parallel_2021}

\subsection{Parallel LR Parsing}
\cite{clarke_error_1993}

\subsection{LR(0) OPG Parsing}
\cite{barenghi_parallel_2015, li_associative_2023}


\section{Semantic Analysis} \label{lit_review_analysis}
\begin{sectionplan}
    \begin{itemize}
        \item Identify what is in semantic anlysis
        \item What occurs during semantic analysis?
        \begin{itemize}
            \item Correcting parse tree if only subset of possible grammar productions are accepted
            \item Symbol table generation
        \end{itemize}
        \item Ways of navigating an AST in parallel
    \end{itemize}
\end{sectionplan}
