\chapter{Literature Review} \label{litreview}

In this literature review I research techniques for creating a parallel
compiler. This review has helped me to develop a better understanding of how
parallel compilers work and how to approach implementing one. Time constraints
make it impossible to review all literature on the topic however
an attempt is made to cover important developments in parallel compilation
research.
\newline \newline
Section \ref{parallelisation_methods} describes hardware capable of executing
code in parallel.
\newline \newline
Section \ref{compiler_parallel_methods} describes different ways of designing a
compiler such that parts of it can be executed simultaneously.
\newline \newline
Section \ref{lit_review_lexing} describes various algorithms to perform lexing
in parallel.
\newline \newline
Section \ref{lit_review_parsing} describes various algorithms to perform parsing
in parallel.
\newline \newline
% Section \ref{lit_review_analysis} describes various algorithms to perform semantic
% analysis in parallel.
% \newline \newline

\section{Parallelization Methods} \label{parallelisation_methods}
This section describes various hardware solutions for executing code in
parallel. Parallel processing methods can strongly impact the performance
for a parallel program as well as the complexity of its implementation. This
makes it worthwhile to research and analyse various forms of parallelization
in order to make informed decisions about how they will be used in the final
implementation.
\newline \newline
Section \ref{simd} describes \gls{simd} instructions which are used to implement
instruction-level parallelism.
\newline \newline
Section \ref{multithreading} describes a more granular form of parallelization
called multi-threading or multiprocessing.
\newline \newline
Section \ref{gpgpu} describes highly parallel computing on graphics cards.

\subsection{Single Instruction Multiple Data} \label{simd}

\gls{simd} instructions are assembly instructions that can process more data
than a typical assembly instruction. Instead of processing data with normal
registers, \gls{simd} instructions use special registers that are several
times larger than normal registers. This allows a processor to compute an
assembly instruction over parts of the register in parallel. These instructions
are also known as vector instructions since they process lists or vectors
of data. As an example, in the Advanced RISC Machines (ARM) instruction
set, the ADD instructions can be contrasted with its corresponding vector
instruction UQADD8 which adds four 8-bit numbers with four other 8-bit numbers
\citep{noauthor_arm_nodate}.

Since vector instructions are functionally similar to a collection of assembly
instructions being executed simultaneously, a program written using vector
instructions can be similar in structure to an equivalent sequential one. This
makes it possible for compilers to automatically utilize these instructions
to speed up existing sequential programs in a process called vectorization.
Vector instructions can also be manually written by the programmer in order to
improve performance \cite{langdale_parsing_2019, mytkowicz_data-parallel_2014}.
Although using vector instructions can be significantly faster than the
sequential case, this performance benefit is only up to a point. Increasing
performance past a certain threshold requires larger registers which take a
long time to architect and standardize in a \gls{cpu}. The methods described in
Section \ref{multithreading} scale better at the cost of having a more complex
implementation.

\subsection{Multithreading/Multiprocessing} \label{multithreading}

Multicore parallelism makes it possible to have multiple programs execute
at the same time through the multi-threading facilities exposed by most
modern \glspl{cpu}. Each process can have its own call stack and its own view
of memory. This freedom in program execution and memory access allows for
performant designs of complex parallel algorithms. The performance of a well
implemented program can potentially scale with the number of \gls{cpu} cores
available to the program. For example, if a parallel program can run in one second
on one thread then in it should run in a quarter of a second on four threads. In
reality, there is overhead from thread creation, synchronization and management
that prevents a perfectly linear performance improvement.

The downside of multithreading is the difficulty in creating reliable, bug free
programs.  This processing model adds new classes of bugs, such as data races
that can occur in shared memory scenarios.  When converting a sequential program
to a multithreaded one, it is usually necessary to re-architect significant
portions of it in order to better fit this processing model. Furthermore,
managing shared memory is difficult and can lead to subtle yet complex bugs.

\subsection{General Purpose Computing on GPUs} \label{gpgpu}

It is possible to use graphics cards for general purpose computing, called
\glspl{gpgpu}. It is conceptually like programming a specialized \gls{cpu} that
has hundreds if not thousands of cores. Programming for a \gls{gpu} requires
a programmer to significantly re-architect a program in order to fit the
\gls{gpu} programming model. There are additional issues, such as the time it
takes to compile \gls{gpu} programs and the large latency between a \gls{gpu}
and main memory. In fact, this latency can be so large that processing small
amounts of data can be much less performant than a single-threaded \gls{cpu}
implementation. In general, large scale computing on a \gls{gpu} can be
strongly worthwhile if program can operate within the constraints of a \gls{gpu}
environment. 

\section{Methods of Parallelizing A Compiler} \label{compiler_parallel_methods}

\cite{gross_parallel_1989} outline three key approaches for parallelizing a
compiler. The first approach consists of splitting the source code of a program
into chunks and processing each chunk individually. Many compilers divide up
source code at the filesystem-level where each file of source code can be
compiled separately from other parts of the program. Although this approach is
already somewhat \gls{data_parallel}, it can be taken even further where each
file is split into even smaller chunks, with each of these chunks being
processed independently.

The second approach is termed computation partitioning. This is described as
a series of sequential computations that can be divided up and processed in
parallel with the results of each computation being joined together at the end.

The final method is pipeling which involves dividing a computation into some
number of phases where each phase depends on the output from the previous phase.

\cite{mark_thierry_vandevoorde_parallel_1988} attempts to implement a parallel
C compiler using a \gls{data_parallel} approach with a traditional a two-pass
compiler architecture.

\section{Lexing} \label{lit_review_lexing}

\cite{scott_programming_2015} describes lexing as the process of removing
unnecessary characters from source code and grouping characters of interest
together into lexemes. This compilation stage emits a list of lexemes that are
later passed to the parser. The purpose of lexing is to simplify the task
of the parser by reducing the size of its input. Lexing is sometimes  called
tokenization or lexical analysis and it is performed by a lexer, sometimes
called a tokenizer or lexical analyser. Lexers are defined using \glspl{reg_exp}
that describe the character sequences which need to be found and emitted from
the lexer as lexemes. Since \glspl{reg_exp} can be described as a \gls{fsm}, a
lexer can similarly be described as one.

A \gls{fsm} is a model of an abstract machine that can be in one of a finite
number of states at any given time. A \gls{fsm} works by accepting input symbols
which can cause the state of the \gls{fsm} to change according to a set of
state transition rules that define the \gls{fsm}. A \gls{fsm} can be described
as either a \gls{nfa} or as a \gls{dfa}. In a parallel lexer where the output
in one thread may depend on the output of another thread, it is undesirable to
have different results based on the same input because it can lead to bugs and
an unpredictable output. This situation can occur in a \gls{nfa} which makes
it non-trivial to design a parallel lexer around \gls{nfa}. The deterministic
property of a \gls{dfa} ensures that the resulting state of a
\gls{dfa} will always be the same for a given input.  As such, structuring a
lexer as a \gls{dfa} can ensure a consistent and repeatable computation for a
given chunk of source code.

A common approach to multithreaded lexing involves structuring the lexer as a
\gls{dfa}, splitting the input into some number of parts, lexing each part on
its own thread and joining up the results at the end. Due to the inherent data
dependency between state transitions in a \gls{dfa}, the initial state for a
chunk of code depends on the output computed from lexing the code just before
it. This causes the issue of determining the initial start state of all the
lexers besides the one computing the first part of the overall input. A solution
to this problem is through speculative simulation where an algorithm speculates
on the unkown initial states of these \gls{dfa}s.

\subsection{Simulation of DFA} \label{simulation_of_dfa}

One approach to speculative simulation is the brute force method of computing
a \gls{dfa} for every possible state it can be initialized with. Such a
lexer begins by splitting its input into chunks and processing each chunk
independently on a separate thread. Once all threads are finished lexing,
the outputs from each thread are checked, starting from the first chunk of
code, such that the finishing state of the lexer corresponds to the initial
state of the lexer that computed the subsequent chunk of code. The correctness
of computing a \gls{dfa} in this way relies on the  associativity of state
transition functions as described in the parallel prefix sum algorithm in
\cite{hillis_data_1986}. This algorithm is explained more formally with both a
message passing and a shared memory implementation by \cite{holub_parallel_2009}
and a variation of it in a cloud computing environment is shown by
\cite{ko_speculative_2012}. The need to compute the output for every possible
initial state can result in many unnecessary computations which can
significantly impact performance for \glspl{dfa} with many states.

\cite{barenghi_parallel_2015} optimize this by using a heuristic where the
source code is split into chunks that start with symbols that have known and
consistent initial states. These symbols are found by scanning ahead when the
source code is being split until such a symbol is found. The symbols that a
chunk of code can start with are language dependant. If a symbol is allowed to
appear in a lexeme, such a string or comment, then multiple initial states must
be additionally computed.

Instead of using a language specific heuristic,
\cite{mytkowicz_data-parallel_2014} attempted to optimize speculative simulation for
any given \gls{dfa}. The core algorithm is very similar to the one previously
described. It begins with the assumption of needing to compute the \gls{dfa} for
every possible initial state the input can be in. Instead of using a language
specific heuristic like \cite{barenghi_parallel_2015}, which is determined ahead
of time, \cite{mytkowicz_data-parallel_2014} define a convergence algorithm
which reduces the number of states that need to be computed at runtime. It
follows from the observation that many states transition to the same state on
a given input symbol. Once this convergence of state occurs, looking up the
state transition for each state becomes redundant as it is going to be the same
for subsequent input symbols. By factoring out these common states, the number
of actual state transitions that need to be computed drops significantly for
most \gls{dfa}s, especially structured and non-adversarial ones. Heuristics
are used to determine when to check for a  convergence of states. The previous
heuristic as well as this convergence algorithm are similar in some ways however
the solution proposed by \cite{mytkowicz_data-parallel_2014} is more generally
applicable by not being tied to a specific language.

\cite{mytkowicz_data-parallel_2014} additionally utilize \gls{simd} instructions
in order to perform the state transitions for a given transition function
and set of states. This parallelizes an important part of the algorithm and
significantly improves performance. A technique called range coalescing is also
used to reduce the number of memory addresses accessed when computing state
transitions for many states. \cite{zhao_--fly_2015} implement a convergence
algorthim similar to the one described by \cite{mytkowicz_data-parallel_2014}.

\subsection{Speculative Simulation} \label{speculative_simulation}

Another method is speculative simulation which attempts to simply guess the
initial state of the \gls{dfa}s, possibly with a way to back track or validate
its result in case of an incorrect guess. \cite{luchaup_multi-byte_2009,
luchaup_speculative_2011} implements a speculative method which simply guesses
the unknown initial \gls{dfa} state. This optimization works due to its specifc
use in intrusion detection systems where a \gls{dfa} spends most of its time
in a small number of states which can be guessed with sufficient accuracy to
outweigh the cost of validation in case of failure.

\subsection{Prescanning} \label{lit_prescanning}

\cite{bernecky_spmdsimd_2003} describes a multi-pass parallel APL tokenizer
written in APL which performs several scanning phases in order to pre-process
source code as well as find strings, comments and identifiers, among other
tokens, in distinct passes over the source code. It is not table driven and is
instead specialized for APL. Since the computational complexity of the compiler
is high and its performance is not evaluated by the authors, it remains unknown
if it is an improved approach to parallel lexing.

A variation of the heuristic approach by \cite{barenghi_parallel_2015} is
implemented by \cite{li_plex_2021} that generalizes it and makes it language
independent. It builds on work from \cite{sinya_simultaneous_2013} and
\cite{zhao_--fly_2015} to perform a pre-scanning phase by generating and
executing a \gls{dfa} based on the lexical grammar that determines the context
for \glspl{dfa} in a subsequent tokenization phase.


% \subsection{Other Approaches} \label{other_approaches}

\begin{comment}
\cite{sinya_simultaneous_2013} propose a novel type of finite automata called
\gls{sfa}.
\newline \newline
Mention \cite{lin_accelerating_2013, wang_hyperscan_2019,
li_plex_2021, asthagiri_associative_1992} Go and look at references cited in
\cite{zhao_--fly_2015}
\newline \newline
Could mention these but the fella doesn't do anything novel (and his
grammar is questionable) \cite{barve_parallel_2014, barve_parallel_2012,
barve_improved_2015}
\end{comment}

\section{Parsing} \label{lit_review_parsing}

Parsing is the second canonical stage of a compiler. The goal of a parser
is to take a list of lexemes from a lexer and structure them as a graph
\citep{scott_programming_2015}. This structure makes it easier to navigate
and manipulate the source code using concepts defined by the language like
variables, functions and control structures. The graph structure created by
a parser is typically called a syntax tree or a parse tree. It is created
according to a parsing grammar that defines how it should be built. The rules
of a parsing grammar that govern how lexemes should be processed in order to
validate an instance of the language and build a parse tree representing it
are called production rules. A parser might additionally process a parse tree
as it is getting built, or during a separate subsequent processing step, which
produces an \gls{ast}. This data structure is later analysed during the semantic
analysis phase. There exist different families of parsers that group parsers by
similarities in their method of parsing.

\gls{ll} parsers read the source code from left to right and construct a parse
tree from the root node down, predicting at each step which production will be
used to expand the current node, based on the next available token of input.
\gls{ll} parsers are also called recursive descent parsers due to their typical
implementation of recursively calling a function for every non-terminal in a
production. Many modern compilers use recursive descent parsers because they
can easily be implemented by hand and make it is easier to generate good error
messages for the user. An alternative approach is to create an \gls{ll} parse
table for a given grammar and use a driver program that parses tokens according
to the rules stored in the parsing table. A program that uses this table to
generate and output source code for a recursive descent parser is a parser
generator.

In \gls{lr} parsing, source code is read left to right but the parse tree is
constructed by first creating the leaf nodes and later grouping these nodes
together into trees. This kind of parser works by maintaining a stack of tokens
and parse tree nodes during parsing. The parser will match this list against
the right side of the grammars production rules. If a match occurs, the tokens
on the stack become children of a new node corresponding the left side of rule.
This new node, corresponding to a subtree of the final parse tree, is pushed
onto the stack. Parsing finishes once the root node (axiom) is built from the
nodes on the stack. This type of parser is also called a shift-reduce parser due
to its two main operations, shifting tokens on the stack and reducing them into
new parse tree nodes.

\gls{ll}($k$) and \gls{lr}($k$) denotes how many tokens these parsers must look
ahead in the input in order to resolve ambiguities between production rules.

\subsection{Parallel LR Parsing} \label{parallel_lr_parsing}

\cite{barenghi_parallel_2015} proposes and implements a non-speculative \gls{lr}
(0) parallel parser generator for \gls{opg}s. A key insight is in constraining
the grammar sufficiently such that it can be deterministically decided whether
a string of bounded length contains the right-hand side of a production
and can be unequivocally replaced by its corresponding left-hand side. This
is in contrast to other approaches that require the parser to speculate on
possible production rules in order to reduce them when parsing in parallel
\citep{mickunas_parallel_1978}.

\cite{li_associative_2023} builds on the work by \cite{barenghi_parallel_2015}
and optimizes the parser for a common case encountered in real word data. When
parsing a long list described by a recursive production rule, the whole list
must be parsed in order to reduce it into a parse tree node. This means that
if the list is large enough to be parsed by multiple threads then the final
operation which reduces the stack will be deferred until all the threads have
finished. Moreover, if there is little work for the parser to do per element
in the list, then the bulk of the parsing will be left until the final joining
phase of the parser. The contribution from \cite{li_associative_2023} is a way
of recognizing operators in the grammar that are associative so that elements of
such lists can be more effectively reduced into parse tree nodes before knowing
they are part of a list.

\begin{comment}
    LLP(q, k) \cite{robin_voetter_parallel_2021}
    \newline \newline
    CYK Parsing \cite{skrzypczak_parallel_nodate}
    \newline \newline
    also mention \cite{mark_thierry_vandevoorde_parallel_1988, alblas_bibliography_1994}
\end{comment}

\section{Semantic Analysis} \label{lit_review_analysis}
% \begin{comment}
\begin{itemize}
    \item Identify what is in semantic anlysis
    \item What occurs during semantic analysis?
    \begin{itemize}
        \item Correcting parse tree if only subset of possible grammar productions are accepted
        \item Symbol table generation
    \end{itemize}
% \end{comment}

\cite{komathukattil_evaluating_nodate} outlines methods of checking whether an
identifier has been used before it is declared.

\cite{seshadri_investigation_1991}
    \item Ways of navigating an AST in parallel
\end{itemize}





