\section{Introduction}
\begin{sectionplan}
     Short explanation of compiler technology and the structure of a compiler.
Explain the compilation process and the use cases for compilation.
\end{sectionplan}

\citep{li_associative_2023}

\subsection{Parallelism In Compilers}
\begin{sectionplan}
     What is meant by parallelism in a compilers implementation? E.g compared to
a compiler generating code that works in parallel.
\end{sectionplan}

\subsection{Benefits of Parallel Compilation}
\begin{sectionplan}
     \begin{itemize}
          \item What is meant by parallelism in a compilers implementation? E.g
                compared to a compiler generating code that works in parallel.
          \item Reasons a parallel compiler implementation is useful.
     \end{itemize}
\end{sectionplan}
\subsection{Parallelisation Methods}
\begin{sectionplan}
     What are the different means of parallelisation? Which ones are being
studied here?
\end{sectionplan}

\subsubsection{Single Instruction Multiple Data (SIMD)}
\begin{sectionplan}
SIMD - hard to use and optimise with. Good for speeding up single threaded
operations *if* a use case is found.
\end{sectionplan}
\subsubsection{Multithreading / Multiprocessing}

\begin{sectionplan}
     Multi-threading / multi-processing - hard to work around in a compiler but
commonly available.
\end{sectionplan}
\subsubsection{General Purpose Computing On GPUs (GPGPU)}

\begin{sectionplan}
     large scale heterogeneous computing (gpu) - Imposes many limitations along
with a very large overhead that makes it useless for smaller - scale workloads.
Availability varies.
\end{sectionplan}

\subsubsection{Very Large Instruction Word (VLIW)}
\begin{sectionplan}
     VLIW can be used to make any program parallel - general purpose VLIW cpu's
not available for this purpose, mostly use in DSP's - esoteric.
\end{sectionplan}

\subsection{Objectives of My Work}

\subsection{Outline of Subsequent Sections}

