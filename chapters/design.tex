\chapter{Design} \label{design}
\begin{comment}
\begin{itemize}
	\item Discuss the various approaches mentioned in the literature review, mentioning
		  their pros and cons. Write about the issues with designing a parallel compiler,
		  e.g issues common to all approaches. Elaborate on why tech choices are being
		  made e.g using rust. 
	\item Discuss my plan / design for the compiler.
\end{itemize}


\end{comment}

\section{Parallelization Method} \label{design_parallel_method}

Computing on the GPU is theoretically the option with the most available
performance. Unfortunately however, programming on a GPU is difficult, running
small tasks on it is impractical and trying to create a compiler that runs
on the GPU requires making big concessions in regard to the compiler
architecture and language design. This leaves a combination of multicore and
\gls{simd} processing which would be most performant in the general case,
however for my \gls{fyp} I intend to only focus on the multithreaded case.
There are other methods that exist today besides the ones I’ve mentioned so far
however they are arguably niche and not commonly available.

\section{Lexer} \label{lexer}

The core issue with lexing in parallel is figuring out the inital state
of the \gls{fsm} for all but the \gls{fsm} processing the first chunk of
input. This can be solved through enumerating through all possible initial
states or simply guessing the most likely state and backtracking if guessed
incorrectly. Speculative approaches that guess the initial \gls{fsm} states
can be effective, as shown by \cite{luchaup_speculative_2011}, however
\cite{mytkowicz_data-parallel_2014} points out two major issues that can
arise. The efficacy of the speculative approach is difficult to predict and
it is limited by the sequential implementation on a single core. In contrast,
enumerating through all possible inital states is predictable against even
randomly generated input and can benefit from fine-grained parallelism afforded
by a single core.

For my implementation I've chosen to use the enumerative approach where I will
enumerate over all possible initial states for the \glspl{fsm}. I hope  this will
lead to a more robust system that is resistant to edge cases and adverserial
input. Although I initially aim to create a handwritten lexer that works in
parallel, I hope to create a table driven lexer if time allows. This will help
me in quickly iterating over the lexical grammar to better fit the parsing
grammar.

\section{Parser} \label{parser}

There are many parser designs available in the literature. The one I have chosen
is the \gls{opg} parser by \cite{barenghi_parallel_2015}. It has a simple design
that allows for a straightforward implementation. 

