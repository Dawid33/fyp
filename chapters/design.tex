\chapter{Design} \label{design}

The design of my compiler is based on the ideas and techniques I encountered
during my literature review. Instead of creating a compiler for a specific
language instance, I intend to implement a table-driven lexer and parser that
will make it easy to implement different kinds of languages. The semantic
analysis phase cannot easily be made generic and will require a distinct
implementation per language. I believe this approach will help me evaluate the
impact of parallel processing on compilation by testing not only my own bespoke
language but also common languages such as JSON or Lua.
\newline\newline
Section \ref{design_lexer} describes the design of the table-driven lexer.
This includes my method of defining lexical grammars, the algorithms used in
pre-processing the lexical grammar and the algorithms used during the parallel
lexical analysis.
\newline\newline
Section \ref{design_parser} describes the design of the table-driven parser.
This includes my method of defining the parsing grammar, its limitations,
the transformation of the grammar into a bounded context grammar and parallel
parsing algorithm.
\newline\newline
Section \ref{design_analyser} describes the general approach for designing the
semantic analyser.

\section{Lexer} \label{design_lexer}

Lexers are defined using \glspl{reg_exp} that describe the character sequences
which need to be found and emitted from the lexer as lexemes. These regular
expressions can be used to construct a \gls{dfa} that accepts the source code
as input and is capable of recognizing whether the input matches any of the
original regular expressions. An example of such a lexical grammar for JSON is
attached in the addendum as listing \ref{lst:json_lexical_grammar}. In order
to construct the \gls{dfa}, it is necessary to convert the regular expressions
into an \gls{nfa} using Thompson's construction algorithm, which results in a
graph where nodes represent the states and the edges represent state transitions
within the \gls{nfa}. This \gls{nfa} can then be converted into a \gls{dfa}
through the powerset construction algorithm.

During this conversion, it is necessary to look for duplicate edges where the
input symbols are the same, but the states being transitioned to are different.
This process enumerates every potential start state of the lexer and solves the
issue posed in section \ref{lit_review_lexing}. These states are recorded and
used during the parallel lexing algorithm.

Sequential lexical analysis can be now be performed by iterating over the
input text, starting at the \glspl{dfa} initial state and following the edges
that correspond to each input symbol. If an edge is found, the state changes
accordingly and the symbol is discarded. This continues until a node is
reached where no edge can be found for the current input symbol. If such a node
represents a terminal of the grammar, that terminal is emitted by the lexer.
If it doesn't, the lexer emits an error. In either case, the lexer's state is
reset to its initial state and the input symbol is processed again instead of
being discarded.

% Insert a figure here like the one in data-parallel finite state machines.

Although it is possible to use the \gls{dfa} directly to perform lexical
analysis, it is more efficient to keep track of the current state in a variable
and compute the state transition by using a lookup table. This state transition
table has the input symbols on one axis and the lexers states on the other axis.
The table can be constructed by traversing the \gls{dfa} from its initial state
and populating it with the state / input symbol pairs represented by edges,
where the state is the superseding state.

In order to lex the source code in parallel, the source code is first split
into roughly equal sized parts and it is inserted into a concurrently accessible
queue. A threadpool of lexers is initialized and each thread takes a part of the
source code and begins to tokenize it. Each thread initialises an instance of
a lexer for every possible initial state the lexer can be in, resulting in as
many outputs as there are possible initial states. In other words, an individual
output is a list of lexemes generated by a lexer when it begins lexing from
some initial state. These outputs are collected in a concurrently accessible
array until all parts of the source code are tokenized. Once the work queue is
empty, the correct output from each part is chosen such that its start state is
the same as the previous chunk's finish state. Remaining outputs are discarded.
This list of token lists, representing the tokenized version of the input source
code, is then passed on to the parser.

The following is a summary of the overall steps that need to be performed before
and during lexical analysis:

\begin{enumerate}
	\item Define the lexical grammar with a mapping between regex's and terminals.
	\item Use that to create NFA graph using Thompson's construction.
	\item Convert the NFA into a DFA using powerset construction.
	\item During generation, look for duplicate state transitions where the
		  characters are the same, but states differ. This indicates that those
		  states are potential start states of the lexer.
	\item Generate a state transition table by traversing the graph.
	\item Split up the source code into $n$ chunks and put them onto a work queue.
	\item These chunks are taken off the work queue by a thread pool. Each chunk
		  is lexed for each possible start state, resulting in $x$ number of outputs for $x$
		  number of start states, per chunk $n$.
	\item Once all chunks have been processed, the correct output from each chunk is
		  chosen such that its start state is the same as the previous chunk's finish
    	  state. Remaining outputs are discarded.
\end{enumerate}

\section{Parser} \label{design_parser}

As mentioned in section \ref{litreview_conclusion}, I will implement
an algorithm for parsing operator precedence grammars as shown by
\cite{barenghi_parallel_2015}. An operator precedence grammar is a kind of
context-free grammar where the right hand-side of each production rule contains
at least one terminal or non-terminal and no right-hand side contains two
consecutive non-terminals. 

In order for it to be possible to parse the grammar in parallel, there must
be a bound on the number of tokens the parser must see in order to reduce
the righthand side of a rule to its left hand side. A grammar that has such a
bound is called a bounded-context grammar, as is described in the literature
\citep{grune_parsing_2008}. \cite{barenghi_parallel_2015} define a set of
restrictions which, when adhered to, ensure that an \gls{opg} is also a
bounded-context grammar with a bound one. They call this \gls{fnf} and it
ensures that the parser is able to parse portions of the code without the
context of what came before or after it.

An \gls{opg} is in \gls{fnf} if and only if an axiom S does not
occur in any right-hand side; no two rules have the same right-hand side;
no rule, possibly except one with the axiom as the left-hand side, has $\epsilon$ as
the right-hand side; renaming rules, i.e., those with a single non-terminal
character as right-hand side, are those and only those with S as the left-hand
side.

Once a grammar is in \gls{fnf}, it can then be parsed by the algorithm
described by \cite{barenghi_parallel_2015}. A traditional linear time parser
for \gls{opg} languages operates on terminals and results in a parse tree of
only terminals. In order to parse this kind of grammar in parallel, a more
general parsing algorithm is presented which allows analysing strings that
contain non-terminals. Implementing this algorithm requires the construction of
an operator precedence table that contains the precedence relationships between
terminals in the grammar. The algorithm to construct such a table is described
by \cite{grune_parsing_2008}.

The input to the parser is prepared similarly to the previous lexing step.
The list of lexemes from the lexer is split into roughly same sized chunks
that are put onto a concurrently accessible queue. A threadpool of parsers is
initialized and begins working through the queue, parsing each chunk of lexemes
and outputting a partial parse tree into a concurrently accessible output array.
These partial parse trees contain the parse tree of the lexemes that could be
reduced, aswell as the internal stack of tokens that could not yet be reduced
due to an insufficient amount of context. Once the queue is empty, the internal
stacks of these partial parse trees are joined and the same parsing algorithm is
used to produce the final parse tree.

The following is a summary of the overall steps that need to be performed before
and during syntax analysis:

\begin{enumerate}
	\item Read grammar from file that defines terminals, non-terminals and
  		  production rules.
	\item Transform grammar into \gls{fnf} as defined in
		  \cite{barenghi_parallel_2015}
	\item Build operator precedence table according to \cite{grune_parsing_2008}
	\item Take the lexer output from the previous step and parse each chunk using
		  the same thread pool approach as the lexer. The parsing algorithm returns a
		  partial parse tree and a stack of lexemes.
	\item Take the outputs from each chunk and join the partial parse trees using
		  the same parsing algorithm. This results in a parse tree of the whole input.
\end{enumerate}

\section{Semantic Analysis} \label{design_analyser}

In order to perform semantic analysis, the parse tree from the previous
step must be transformed in an \gls{ast}. This can be done by traversing the
\gls{ast} in post-order depth first traversal, reducing leaf nodes into ast nodes,
thereby producing the \gls{ast} bottom-up. This can be done in linear time as
the parse tree resulting from the parsing step can be stored as an array whereby
traversing the array from start to finish equates to a post-order traversal of
the tree. 

Semantic analysis is performed by traversing the \gls{ast} with a
pre-order depth first traversal and keeping a record of identifiers
and their metadata in a symbol table. Parallel semantic analysis can be
performed by splitting the work based on nested scopes in the \gls{ast}
\citep{seshadri_investigation_1991}. If during a pre-order depth first traversal
of the \gls{ast} a new scope is encountered, the analyser starts a new thread
for parsing that inner scope while passing to it the symbol table it has
constructed thus far. Working from the top to bottom of the \gls{ast} this way
ensures that the whole of the \gls{ast} is analysed correctly. The drawback with
this approach is that the efficacy of this parallelism is limited to the number
and size of the nested scopes, which additionally makes splitting the work
equally between threads a challenge.

Summary of the overall steps that need to be performed during semantic
analysis:

\begin{enumerate}
	\item Iterate over the parse tree in post-order DFS in order to generate the Abstract syntax tree.
	\item Iterate over AST using pre-order DFS, creating a new thread and symbol table for certain nodes.
	\item Check if variables are defined before they are used by checking if they exist in ancestor symbol tables.
\end{enumerate}



