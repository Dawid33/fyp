\chapter{Conclusion} \label{conclusion} 

\section{Summary}

This FYP began with an introduction to the fundamentals of sequential and parallel compilation,
highlighting the traditional focus on minimizing memory use and optimizing for single-threaded
performance, and juxtaposes this with the contemporary shift towards parallel processing methods to
exploit the capabilities of multi-core processors. I elaborated on various parallelization methods,
including \gls{simd}, multithreading, \gls{gpgpu}  providing examples of their applications in
compiler design.

A significant portion of this FYP delved into a literature review on methods for parallelizing
compilers, focusing on lexing, parsing, and semantic analysis. I examined existing techniques, their
potential for improving compiler performance, and their implications for language design. The review
covers topics such as deterministic and non-deterministic finite automata, speculative simulation
for lexing, parallel LR and LL parsing algorithms, and the challenges of semantic analysis in a
parallelized context.

Following the literature review, I outlined the design and implementation strategies for developing
a parallel compiler, including considerations for lexing, parsing, and semantic analysis. The
document discusses the practical aspects of constructing a parallel compiler, emphasizing the
need for carefully considering the compiler architecture such that it can efficiently handle the
complexities of parallel processing.

\section{Discussion}

\begin{listing}[H]
\begin{minted}[linenos, breaklines=true, fontsize=\scriptsize]{text}
CHAR = "[a-zA-Z]"
BOOL = "true|false"
NUMBER = "[0-9][0-9]*"
WHITESPACE = "( |\n|\t|\r)*"
LBRACE = "\{"
RBRACE = "\}"
LSQUARE = "\["
RSQUARE = "\]"
COMMA = ","
COLON = ":"
QUOTES = "\""
\end{minted}
\caption{JSON lexical grammar keywords and their corresponding regular expressions}
\label{lst:json_lexical_grammar}
\end{listing}

\begin{listing}[H]
\begin{minted}[linenos, breaklines=true, fontsize=\scriptsize]{text}
%nonterminal OBJECT
%nonterminal MEMBERS
%nonterminal PAIR
%nonterminal VALUE
%nonterminal STRING
%nonterminal CHARS
%nonterminal ARRAY
%nonterminal ELEMENTS

%axiom OBJECT

%terminal LBRACE
%terminal RBRACE
%terminal LSQUARE
%terminal RSQUARE
%terminal COMMA
%terminal COLON
%terminal BOOL
%terminal QUOTES
%terminal CHAR
%terminal NUMBER

%%

OBJECT : LBRACE RBRACE
       | LBRACE MEMBERS RBRACE
       ;

MEMBERS : PAIR
        | PAIR COMMA MEMBERS
        ;

PAIR : STRING COLON VALUE
     ;

VALUE : STRING
      | NUMBER
      | OBJECT
      | ARRAY
      | BOOL
      ;

STRING : QUOTES QUOTES
       | QUOTES CHARS QUOTES
       ;

CHARS : CHAR
      | CHAR CHARS
      ;

ARRAY : LSQUARE RSQUARE
      | LSQUARE ELEMENTS RSQUARE
      ;

ELEMENTS : VALUE
         | VALUE COMMA ELEMENTS
         ;
\end{minted}
\caption{An operator precedence parsing grammar for JSON.}
\label{lst:json_grammar}
\end{listing}


\begin{longlisting}
\begin{minted}[linenos, breaklines=true, fontsize=\scriptsize]{text}
%nonterminal chunk
%nonterminal statList
%nonterminal stat
%nonterminal elseIfBlock
%nonterminal exprThenElseIfB
%nonterminal exprThen
%nonterminal name
%nonterminal retStat
%nonterminal label
%nonterminal funcName
%nonterminal nameDotList
%nonterminal varList
%nonterminal var
%nonterminal nameList
%nonterminal exprList
%nonterminal expr
%nonterminal logicalOrExp
%nonterminal logicalAndExp
%nonterminal relationalExp
%nonterminal concatExp
%nonterminal additiveExp
%nonterminal multiplicativeExp
%nonterminal unaryExp
%nonterminal caretExp
%nonterminal baseExp
%nonterminal prefixExp
%nonterminal functionCall
%nonterminal functionDef
%nonterminal parList
%nonterminal tableConstructor
%nonterminal fieldList
%nonterminal fieldListBody
%nonterminal field
%nonterminal typeExpr
%nonterminal ptrTypeStart

%axiom chunk

%terminal ENDFILE
%terminal RETURN
%terminal SEMI
%terminal COLON
%terminal COLON2
%terminal DOT
%terminal DOT3
%terminal COMMA
%terminal LBRACK
%terminal RBRACK
%terminal LBRACE
%terminal RBRACE
%terminal LPAREN
%terminal RPAREN
%terminal EQ
%terminal BREAK
%terminal GOTO
%terminal DO
%terminal END
%terminal WHILE
%terminal REPEAT
%terminal UNTIL
%terminal IF
%terminal THEN
%terminal ELSEIF
%terminal ELSE
%terminal FOR
%terminal IN
%terminal FUNCTION
%terminal LET
%terminal NIL
%terminal FALSE
%terminal TRUE
%terminal NUMBER
%terminal STRING
%terminal NAME
%terminal PLUS
%terminal MINUS
%terminal ASTERISK
%terminal DIVIDE
%terminal CARET
%terminal PERCENT
%terminal DOT2
%terminal LT
%terminal GT
%terminal LTEQ
%terminal GTEQ
%terminal EQDOUBLE
%terminal NEQ
%terminal AND
%terminal OR
%terminal NOT
%terminal UMINUS
%terminal SHARP
%terminal SEMIFIELD
%terminal EQ
%terminal QUESTIONMARK
%terminal STRUCT
%terminal COMMENT

%%

chunk : statList
	;

statList : stat
	| SEMI
	| stat SEMI
	| statList SEMI stat
	| statList SEMI
	;

stat : baseExp EQ expr
	| functionCall
	| retStat
	| LBRACE statList RBRACE
	| LBRACE RBRACE
	| WHILE expr LBRACE statList RBRACE
	| WHILE expr LBRACE RBRACE
	| IF exprThen RBRACE
	| IF exprThen RBRACE elseIfBlock
	| STRUCT baseExp LBRACE RBRACE
	| STRUCT baseExp LBRACE fieldList RBRACE
	| FUNCTION baseExp LBRACK RBRACK LBRACE statList RBRACE
	| FUNCTION baseExp LBRACK fieldList RBRACK LBRACE statList RBRACE
	| FUNCTION baseExp LBRACK fieldList RBRACK LBRACE RBRACE
	| FUNCTION baseExp LBRACK baseExp RBRACK LBRACE statList RBRACE
	| FUNCTION baseExp LBRACK baseExp RBRACK LBRACE RBRACE
	| FUNCTION baseExp LBRACK RBRACK LBRACE RBRACE
	| FOR nameList IN exprList LBRACE statList RBRACE
	| FOR nameList IN exprList LBRACE RBRACE
	| LET baseExp EQ expr
	| LET baseExp
	;

functionCall : baseExp LPAREN exprList RPAREN
	| baseExp LPAREN expr RPAREN
	| baseExp LPAREN RPAREN
	;

retStat : RETURN SEMI
	| RETURN exprList SEMI
	| RETURN expr SEMI
	| RETURN
	| RETURN exprList
	| RETURN expr
	;

elseIfBlock : ELSEIF expr LBRACE statList RBRACE
	| ELSEIF expr LBRACE RBRACE
	| ELSEIF expr LBRACE RBRACE elseIfBlock
	| ELSEIF expr LBRACE statList RBRACE elseIfBlock
	| ELSE LBRACE RBRACE
	| ELSE LBRACE statList RBRACE
	;

exprThen : expr LBRACE statList
	| expr LBRACE
	;

exprList : expr COMMA expr
	| exprList COMMA expr
	;

expr : logicalOrExp
	;

logicalOrExp : logicalAndExp
	| logicalOrExp OR logicalAndExp
	;

logicalAndExp : relationalExp
	| logicalAndExp AND relationalExp
	;

relationalExp : concatExp
	| relationalExp LT concatExp
	| relationalExp GT concatExp
	| relationalExp LTEQ concatExp
	| relationalExp GTEQ concatExp
	| relationalExp NEQ concatExp
	| relationalExp EQDOUBLE concatExp
	;

concatExp : additiveExp
	| additiveExp DOT2 concatExp
	;

additiveExp : multiplicativeExp
	| additiveExp PLUS multiplicativeExp
	| additiveExp MINUS multiplicativeExp
	;

multiplicativeExp : unaryExp
	| multiplicativeExp ASTERISK unaryExp
	| multiplicativeExp DIVIDE unaryExp
	| multiplicativeExp PERCENT unaryExp
	;

unaryExp : caretExp
	| NOT unaryExp
	| SHARP unaryExp
	| UMINUS unaryExp
	;

caretExp : baseExp
	| baseExp CARET caretExp
	;

baseExp : NIL
	| FALSE
	| TRUE
	| NUMBER
	| STRING
	| NAME
	| functionDef
	| prefixExp
	;

prefixExp : var
	| functionCall
	| LPAREN expr RPAREN
	;

fieldList : fieldListBody
	| fieldListBody COMMA
	;

fieldListBody : field
	| fieldListBody COMMA field
	;

field : baseExp COLON baseExp
	;

var : prefixExp DOT baseExp
	;

varList : var COMMA var
	| varList COMMA var
	;

funcName : nameDotList
	| nameDotList COLON baseExp
	;

nameDotList : baseExp DOT baseExp
	| nameDotList DOT baseExp
	;
\end{minted}
\caption{An operator precedence parsing grammar for the test language.}
\label{lst:test_grammar}
\end{longlisting}
