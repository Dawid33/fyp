
\makeglossaries

\setulcolor{lightgray}
\renewcommand*{\glstextformat}[1]{\textcolor{black}{%
\ul{\mbox{#1}}}}

\newglossaryentry{compiler}
{
    name=compiler,
    description={
    A compiler is a program which translates a high-level
programming language into a semantically equivalent program written in a
lower-level programming language such as assembly or machine language}
}

\newglossaryentry{strongly_parallel}
{
    name=strongly parallel,
    description={ TODO }
}

\newglossaryentry{compiler_frontend}
{
    name=compiler frontend,
    description={The first few stages of compilation including tokenisation,
parsing and semantic analysis are colloquially called the frontend of a compiler}
}

\newglossaryentry{linker}
{
    name=linker,
    description={TODO}
}

\newglossaryentry{data_parallel}
{
    name=data-parallel,
    description={Data parallelism is parallelization across multiple processors
in parallel computing environments. It focuses on distributing the data across
different processors which operate on the data in parallel. Importantly, it
involves running the same task on different components of the data}
}

\newglossaryentry{rustc}
{
    name=rustc,
    description={The first and most popular compiler for the Rust programming
language that is being developed by the Rust Foundation}
}

\newglossaryentry{network_packet_parsing} {
    name=network packet parsing,
    description={An algorithm that reads and analyzes network packets in order
to decide how they should be processed in a network-connected device. A network
packet is a formatted unit of data carried throught out a computer network}
}

\newglossaryentry{huffman_coding}
{
    name=huffman coding,
    description={A Huffman code is a type of optimal prefix code that is
commonly used for lossless data compression. The process of finding or using
such a code is Huffman coding}
}

\DeclareAcronym{gpgpu}{
  short = GPGPU,
  long  = General Purpose Graphical Processing Unit Computation,
  first-style=long-short
}

\DeclareAcronym{fyp}{
  short = FYP,
  long  = Final Year Project,
  first-style=long-short
}

\DeclareAcronym{simd}{
  short = SIMD,
  long  = Single Instruction Multiple Data,
  first-style=long-short
}

\DeclareAcronym{dpu}{
  short = DPU,
  long  = Data Processing Unit,
  first-style=long-short
}

\DeclareAcronym{cpu}{
  short = CPU,
  long  = Central Processing Unit,
  first-style=long-short
}

\DeclareAcronym{gpu}{
  short = GPU,
  long  = Graphics Processing Unit,
  first-style=long-short
}

\DeclareAcronym{ram}{
  short = RAM,
  long  = Random Access Memory,
  first-style=long-short
}
