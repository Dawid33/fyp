
\makeglossaries

\setulcolor{lightgray}
\renewcommand*{\glstextformat}[1]{\textcolor{black}{%
\mbox{#1}}}

\newglossaryentry{compiler}
{
    name=compiler,
    description={software that converts a program written in a high-level
language into semantically equivalent instructions in a low-level programming
language, the resulting instructions can then be read and executed by the
computer}
}

\newglossaryentry{compiler_frontend}
{
    name=compiler frontend,
    description={the initial stages of compilation are colloquially referred
to as the frontend of a compiler, such stages include tokenisation, parsing and
semantic analysis}
}

\newglossaryentry{linker}
{
    name=linker,
    description={
	a software tool that performs the final step of creating an
executable program known as linking. This step involves combining object files
created by a compiler and resolving external symbol references
	}
}

\newglossaryentry{data_parallel}
{
    name=data-parallel,
    description={in computer processing, data parallelism refers to dividing or partitioning the
data amongst multiple processors (aka graphical processing units or nodes), so
that multiple threads can operate concurrently (applying the same operation) on
each segment, thereby increasing data throughput}
}

\newglossaryentry{reg_exp}
{
    name=regular expression,
    description={A regular expression is a sequence of characters that specifies
a match pattern in text}
}

\newglossaryentry{rustc}
{
    name=rustc,
    description={the first and most popular compiler for the Rust programming
language that is being developed by the Rust Foundation}
}

\newglossaryentry{network_packet_parsing} {
    name=network packet parsing,
    description={a algorithm that reads and analyzes network packets in order
to decide how they should be processed in a network-connected device. A network
packet is a formatted unit of data carried throughtout a computer network}
}

\newglossaryentry{huffman_coding}
{
    name=huffman encoding,
    description={an algorithm used to compress information (characters,
images, audio and so forth) without losing information (i.e., a lossless data
compression algorithm), named after its creator David Huffman in 1952. The
algorithm works by identifying frequently occurring items in the information
and assigning them a short sequence of bits in the encoding, while using longer
sequences of bits for less frequently occurring items}
}

\newglossaryentry{fsmg}{
	name={finite state machine},
    description={
		a mathematical model used to represent systems with distinct states and
transitions between those states. It consists of a set of states, a set of
transitions, and an initial state. At any given time, the system is in one of
its defined states. Transitions between states occur in response to specific
inputs, and these transitions are governed by a set of rules or conditions
	}
}

\newglossaryentry{opgg}{
	name={operator precedence grammar},
    description={TODO}
}

\newglossaryentry{astg}{
	name={abstract syntax tree},
    description={
		a hierarchical, tree-like data structure that represents the syntactic structure
of source code in a programming language, excluding details of the concrete
syntax like punctuation or whitespace. Each node in the tree corresponds to
a syntactic construct in the source code, such as expressions, statements, or
declarations. The tree's structure reflects the grammatical relationships and
nesting of these constructs, with parent-child relationships indicating how they
are related in the code}
}

%%
%% ACRONYMS
%%

\newglossaryentry{dky}{
	type = \acronymtype, 
	name = {DKY},
	description = {Doesn't Know Yet},
	first = {Doesn't Know Yet (DKY)},
}

\newglossaryentry{json}{
	type = \acronymtype, 
	name = {JSON},
	description = {JavaScript Object Notation},
	first = {JavaScript Object Notation (JSON)},
}

\newglossaryentry{amd}{
	type = \acronymtype, 
	name = {AMD},
	description = {Advanced Micro Devices},
	first = {Advanced Micro Devices (AMD)},
}

\newglossaryentry{ast}{
	type = \acronymtype, 
	name = {AST},
	description = {Abstract Syntax Tree},
	first = {Abstract Syntax Tree (AST)\glsadd{astg}},
	see = [Glossary:]{astg}
}

\newglossaryentry{opg}{
	type = \acronymtype, 
	name = {OPG},
	description = {Operator Precedence Grammar},
	first = {Operator Precedence Grammar (OPG)\glsadd{opgg}},
	see = [Glossary:]{opgg}
}

\newglossaryentry{lr}{
	type = \acronymtype, 
	name = {LR},
	description = {Left-to-right, Right-most-derivation},
	first = {Left-to-right, Right-most-derivation (LR)},
}

\newglossaryentry{ll}{
	type = \acronymtype, 
	name = {LL},
	description = {Left-to-right, Left-most-derivation},
	first = {Left-to-right, Left-most-derivation (LL)},
}

\newglossaryentry{fsm}{
	type = \acronymtype, 
	name = {FSM},
	description = {Finite State Machine},
	first = {Finite State Machine (FSM)\glsadd{fsmg}},
	see = [Glossary:]{fsmg}
}

\newglossaryentry{fyp}{
	type=\acronymtype, 
	name = {FYP},
	description  = {Final Year Project},
	first={Final Year Project (FYP)},
}


\newglossaryentry{gpgpu}{
	type=\acronymtype, 
	name = {GPGPU},
	description = {General Purpose Graphical Processing Unit},
	first={General Purpose Graphical Processing Unit (GPGPU)},
}

\newglossaryentry{simd}{
	type=\acronymtype, 
  name = {SIMD},
  description  = {Single Instruction Multiple Data},
	first={Single Instruction Multiple Data (SIMD)},
}

\newglossaryentry{dpu}{
	type=\acronymtype, 
  name = {DPU},
  description  = {Data Processing Unit},
	first={Data Processing Unit (DPU)},
}

\newglossaryentry{cpu}{
	type=\acronymtype, 
  name = {CPU},
  description  = {Central Processing Unit},
	first={Central Processing Unit (CPU)},
}

\newglossaryentry{gpu}{
	type=\acronymtype, 
  name = {GPU},
  description  = {Graphics Processing Unit},
	first={Graphics Processing Unit (GPU)},
}

\newglossaryentry{ram}{
	type=\acronymtype, 
  name = {RAM},
  description  = {Random Access Memory},
	first={Random Access Memory (RAM)},
}

\newglossaryentry{nic}{
	type=\acronymtype, 
  name = {NIC},
  description  = {Network Interface Controller},
	first={Network Interface Controller (NIC)},
}

\newglossaryentry{rocm}{
	type=\acronymtype, 
  name = {ROCm},
  description  = {Radeon Open Compute platforM},
	first={Radeon Open Compute platforM (ROCm)},
}

\newglossaryentry{nfa}{
	type=\acronymtype, 
  name = {NFA},
  description  = {Non-deterministic Finite Automata},
	first={Non-deterministic Finite Automata (NFA)},
}

\newglossaryentry{dfa}{
	type=\acronymtype, 
  name = {DFA},
  description  = {Deterministic Finite Automata},
	first={Deterministic Finite Automata (DFA)},
}

\newglossaryentry{sfa}{
	type=\acronymtype, 
	name = {SFA},
	description = {Simultaneous Finite Automata},
	first={Simultaneous Finite Automata (SFA)},
}


