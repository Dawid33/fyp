
\makeglossaries

\setulcolor{lightgray}
\renewcommand*{\glstextformat}[1]{\textcolor{black}{%
\ul{\mbox{#1}}}}

\newglossaryentry{compiler}
{
    name=compiler,
    description={
    A compiler is a program which translates a high-level
programming language into a semantically equivalent program written in a
lower-level programming language such as assembly or machine language}
}

\newglossaryentry{compiler_frontend}
{
    name=compiler frontend,
    description={The first few stages of compilation including tokenisation,
parsing and semantic analysis are colloquially called the frontend of a compiler}
}

\newglossaryentry{linker}
{
    name=linker,
    description={TODO}
}

\newglossaryentry{data_parallel}
{
    name=data-parallel,
    description={Data parallelism is parallelization across multiple processors
in parallel computing environments. It focuses on distributing the data across
different processors which operate on the data in parallel. Importantly, it
involves running the same task on different components of the data}
}

\newglossaryentry{reg_exp}
{
    name=regular expression,
    description={A regular expression is a sequence of characters that specifies
a match pattern in text}
}

\newglossaryentry{rustc}
{
    name=rustc,
    description={The first and most popular compiler for the Rust programming
language that is being developed by the Rust Foundation}
}

\newglossaryentry{network_packet_parsing} {
    name=network packet parsing,
    description={An algorithm that reads and analyzes network packets in order
to decide how they should be processed in a network-connected device. A network
packet is a formatted unit of data carried throught out a computer network}
}

\newglossaryentry{huffman_coding}
{
    name=huffman coding,
    description={A Huffman code is a type of optimal prefix code that is
commonly used for lossless data compression. The process of finding or using
such a code is Huffman coding}
}

\newglossaryentry{fsmg}{
	name={finite state machine},
    description={}
}


%%
%% ACRONYMS
%%

\newglossaryentry{fsm}{
	type = \acronymtype, 
	name = {FSM},
	description = {Finite State Machine},
	first = {Finite State Machine (FSM)\glsadd{fsmg}},
	see = [Glossary:]{fsmg}
}

\newglossaryentry{fyp}{
	type=\acronymtype, 
	name = {FYP},
	description  = {Final Year Project},
	first={Final Year Project (FYP)},
}


\newglossaryentry{gpgpu}{
	type=\acronymtype, 
	name = {GPGPU},
	description = {General Purpose Graphical Processing Unit Computation},
	first={General Purpose Graphical Processing Unit Computation (GPGPU)},
}

\newglossaryentry{simd}{
	type=\acronymtype, 
  name = {SIMD},
  description  = {Single Instruction Multiple Data},
	first={Single Instruction Multiple Data (SIMD)},
}

\newglossaryentry{dpu}{
	type=\acronymtype, 
  name = {DPU},
  description  = {Data Processing Unit},
	first={Data Processing Unit (DPU)},
}

\newglossaryentry{cpu}{
	type=\acronymtype, 
  name = {CPU},
  description  = {Central Processing Unit},
	first={Central Processing Unit (CPU)},
}

\newglossaryentry{gpu}{
	type=\acronymtype, 
  name = {GPU},
  description  = {Graphics Processing Unit},
	first={Graphics Processing Unit (GPU)},
}

\newglossaryentry{ram}{
	type=\acronymtype, 
  name = {RAM},
  description  = {Random Access Memory},
	first={Random Access Memory (RAM)},
}

\newglossaryentry{nic}{
	type=\acronymtype, 
  name = {NIC},
  description  = {Network Interface Controller},
	first={Network Interface Controller (NIC)},
}

\newglossaryentry{rocm}{
	type=\acronymtype, 
  name = {Rocm},
  description  = {Radeon Open Compute platforM},
	first={Radeon Open Compute platforM (Rocm)},
}

\newglossaryentry{nfa}{
	type=\acronymtype, 
  name = {NFA},
  description  = {Non-deterministic Finite Automata},
	first={Non-deterministic Finite Automata (NFA)},
}

\newglossaryentry{dfa}{
	type=\acronymtype, 
  name = {DFA},
  description  = {Deterministic Finite Automata},
	first={Deterministic Finite Automata (DFA)},
}

\newglossaryentry{sfa}{
	type=\acronymtype, 
  name = {SFA},
  description  = {Simultaneous Finite Automata},
	first={Simultaneous Finite Automata (SFA)},
}
